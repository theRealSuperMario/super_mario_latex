\documentclass[a4paper]{article}

%% Language and font encodings
\usepackage[english]{babel}
\usepackage[utf8x]{inputenc}
\usepackage[T1]{fontenc}
\usepackage{amsmath}
\usepackage{super_math}

%% Sets page size and margins
\usepackage[a4paper,top=3cm,bottom=2cm,left=3cm,right=3cm,marginparwidth=1.75cm]{geometry}

%% Useful packages
\usepackage{amsmath}
\usepackage{amsthm}
\usepackage{amssymb}
\usepackage{graphicx}
\usepackage[colorinlistoftodos]{todonotes}
\usepackage[colorlinks=true, allcolors=blue]{hyperref}
\usepackage{neuralnetwork}
\usepackage{sidecap}

\newtheorem{theorem}{Theorem}[section]
\newtheorem{corollary}{Corollary}[theorem]
\newtheorem{lemma}[theorem]{Lemma}



\title{Command Overview}
\author{theRealSuperMario}


%%% --- commands ---

%network notation with nonlinearities
\usepackage{siunitx}
\usepackage{pdflscape}


\begin{document}
\maketitle
\begin{abstract}
\end{abstract}

\begin{landscape}
\thispagestyle{empty}
\begin{table}  
	\fontsize{7}{9}\selectfont
	% \tiny
	\begin{tabular}{l | c | c | c }
		-		& Befehl in der Vorlage 		& Wirkung 			& Anmerkungen 	\\
		\hline
		\textbf{Konventionen zur Darstellung}	&	-		&		-		&	-	\\
		Zahlen mit Einheit	& \verb| \SI{1,234}{\milli\gram\per\tesla} |	& $ \SI{1,234}{\milli\gram\per\tesla} $	&	 	\\
		Zahlen ohne Einheit	& \verb|\num{8}|	&	$\num{8}$		& beispielsweise für Blendenzahl	\\
		Abkürzungen (Einfügen eines kleinen Abstandes)	& \verb| z.\,B. |	& z.\,B.		&		\\
		
		\textbf{Mathematische Funktionen} &			&				&			\\
		trigonometrische Funktionen  	& \verb| \sin(x), \cos(x), \tan(x) |	& $\sin(x), \cos(x), \tan(x)$	& Schrift wird aufrecht gesetzt \\
		
		\textbf{Lineare Algebra aka Matrizen} &			&				&			\\
		Vektoren, Matrizen --> \textbf{fett}	& \verb|\vec{A}|	& $\vec{A}$			& Im Endeffekt nur \verb|\boldsymbol{A}|	\\
		Einheitsmatrix			& \verb| \II |	&	$\II$	& mit DeclarMathOperator erzeugt. \\
		transponierte Matrix		& \verb| \vec{A}^\TT |	&	$\vec{A}^\TT$	& im Enddeffekt nur \verb|\mathrm{T}|	\\    
		Norm eines Vektors		& \verb| \norm{x} |		&	$\norm{x}$ & - \\
		
		\textbf{Integrale}	& 	& 	&  \\
		1D- uneigentliche Integrale 	& \verb+ \infint f(x) \dd x +	& $ \infint f(x) \dd x $	& den \verb|\dd| Befehl nicht vergessen. \\
		1D - uneigentliches Integral Typ 2	& \verb| \uint{f(x)} \dd x |	& $ \uint{f(x)} \dd x $ 	& - \\
		1D - Integral mit oberer und unterer Grenze	& \verb| \lint{x}{a}{b}{f(x)} \dd x |		& $ \lint{x}{a}{b}{f(x)} \dd x $	&  \\
		
		\textbf{Summen} &  &  &  \\
		Reihe von $k=-\infty$ bis $\infty$	&	\verb| \usum{k}{a_k x^k} |	&	$\usum{k}{a_k x^k}$	& \\
		Summe von k= a bis b			&	\verb| \lsum{k}{1}{\infty}{ \frac{1}{k^2} } |	& $ \lsum{k}{1}{\infty}{ \frac{1}{k^2} } $ 	& \\
		
		Index für Abtastung	& \verb| f_\abtast |	& $f_\abtast$	& im Endeffekt \verb|\mathrm{A}|	\\
		Erwartungswert		& \verb| \E{x} |			&	$\E{x}$	&	\\
		
		\textbf{Mathematische Zeichen und Symbole}	& & &\\
		imaginäre Einheit, eulersche Zahl & \verb| \ii, \jj, \ee |	&      $\ii, \jj, \ee $ & 	\\ 
		Real und Imaginärteil	& \verb|\Re{s}, \Im{s}|		& 	$\Real{s}, \Imag{s}$ & \\
		Integrale 1D, 2D, 3D & \verb|\int, \iint, \iiint|	&	$\int, \iint, \iiint$ & \\
		Differenziale beim Integral	&	\verb| \dd x |	&	$ \dd x $	& \\
		Zahlenmengen R,N,Z,C,Q		&	\verb| \rz, \nz, \gz, \cz, \qz |	&	$\rz, \nz, \gz, \cz, \qz$ & \\
		(soll gelten)		&	\verb| \sbe |	& $a \sbe b$	& \\
		entspricht		&	\verb| \entspr |	& $a \entspr b $	&	\\
		\verb|10^something|			&	\verb| \pow{2} |	& $ \pow{2} $	&	\\
		
		\textbf{Transformationen} & & &\\
		Fourier-, Laplace,- Z-Transformation: Mathematikschrift mathcal		& \verb| \FT, \iFT|	& $\FT$ , ~~  $\iFT$& \\
		Doetschsymbol Zeitbereich - Laplace-/Fourier-/Z-bereich horizontal	& \verb|f(t) \TZ F(f)|			& $f(t) \TZ F(f)$	& \\
		Doetschsymbol Laplace-/Fourier-/Z-bereich - Zeitbereich horizontal	& \verb|F(f) \ZT f(t)|		& $F(f) \ZT f(t) $	& \\
		
		%Doetschsymbol Zeitbereich - Laplace-/Fourier-/Z-bereich vertikal	& \verb|f(t) \vTZ F(f)|			& $f(t) \vTZ F(f)$	& \\
		%Doetschsymbol Laplace-/Fourier-/Z-bereich - Zeitbereich vertikal	& \verb|F(f) \vZT f(t)|			& $F(f) \vZT f(t) $	& \\
		
		Doetschsymbol 2D Zeitbereich - Laplace-/Fourier-/Z-bereich horizontal	& \verb| \TZz | 		& $\TZz$	& analog mit \verb| \ZTz, \vTZz, \vZTz | \\
		
	\end{tabular}
\end{table}  
\end{landscape}
\end{document}